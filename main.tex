\documentclass[11pt,a4paper]{article}

% --- Езици и кодировки ---
\usepackage[utf8]{inputenc}
\usepackage[T2A]{fontenc}
\usepackage[bulgarian]{babel}

% --- Стил и оформление ---
\usepackage{geometry}
\geometry{margin=2.54cm} % стандартни полета 1 inch

\usepackage{parskip} % разстояние между абзаци, без отстъп
\setlength{\parindent}{0pt} % нулев отстъп

\usepackage{microtype} % по-добра типография
\usepackage{hyperref}  % за линкове
\hypersetup{
    colorlinks=true,
    linkcolor=blue,
    urlcolor=blue,
    pdftitle={Биткойн: Директна електронна парична система},
    pdfauthor={Сатоши Накамото}
}

% --- Страници ---
\usepackage{fancyhdr}
\pagestyle{fancy}
\fancyhf{}
\fancyfoot[C]{\thepage}

% --- Заглавие ---
\title{\textbf{Биткойн: Директна електронна парична система}}
\author{Сатоши Накамото}
\date{}

\begin{document}

\maketitle

\begin{abstract}
Чистa версия от равноправни участници (peer-to-peer) на електронни пари би позволила онлайн плащанията да се изпращат директно от една страна към друга, без да се преминава през финансова институция. Цифровите подписи осигуряват част от решението, но основните предимства се губят, ако все още е необходима доверена трета страна, за да се предотврати двойното харчене. Предлагаме решение на проблема с двойното харчене, използвайки мрежа с директна връзка. Мрежата поставя времеви печат на транзакциите, като ги хешира в непрекъсната верига от базирано на хеш доказателство за работа, образувайки запис, който не може да бъде променен без повторно извършване на доказателството за работа. Най-дългата верига служи не само като доказателство за последователността от събития, на които сме свидетели, но и като доказателство, че е дошла от най-големия пул от процесорна мощност. Докато по-голямата част от процесорната мощност се контролира от възли, които не си сътрудничат, за да атакуват мрежата, те ще генерират най-дългата верига и ще изпреварват нападателите. Самата мрежа изисква минимална структура. Съобщенията се излъчват на база „най-добри усилия“ и възлите могат да напускат и да се присъединяват отново към мрежата по желание, приемайки най-дългата верига от доказателства за работа (proof-of-work) като доказателство за това какво се е случило, докато са отсъствали.
\end{abstract}

\section{Въведение}

В интернет-базираната търговия доверието е основа на всяка транзакция. Почти всички разплащания разчитат на доверена трета страна, като банки или платежни системи, за да обработят транзакциите. Въпреки че системата работи достатъчно добре при повечето случаи, тя все пак страда от слабости, присъщи на модела, базиран на доверие...

% (тук продължаваш с превода)

\end{document}